\documentclass[10pt,letterpaper]{article}

\usepackage[pdftex]{graphicx}
\usepackage{amsmath} % just math
\usepackage{amssymb} % allow blackboard bold (aka N,R,Q sets)
\usepackage{amsmath,amsthm}
\usepackage{mathtools}
\linespread{1.6}  % double spaces lines
\usepackage[left=1in,top=1in,right=1in,bottom=1in,nohead]{geometry}
\usepackage{cancel}
\usepackage{textcomp}
\usepackage{longtable}
\usepackage{rotating}
\allowdisplaybreaks
\usepackage{scrextend}
\usepackage{listings}
\usepackage{enumerate}
\usepackage{algorithm}
\usepackage{algorithmic}
\usepackage[usenames,dvipsnames]{color}

\newcommand{\partialderivative}[2]{\frac{\partial#1}{\partial#2}}

\begin{document}

\linespread{1} % single spaces lines
\small \normalsize %% dumb, but have to do this for the prev to work
%%%%%%%%%%%%%%%%%%%%%%%%%%%%%%%%%%%%%%%%%%%%%%%%%%%%%%%%%%%%%%%%%%%%%%%
%% Title
%%%%%%%%%%%%%%%%%%%%%%%%%%%%%%%%%%%%%%%%%%%%%%%%%%%%%%%%%%%%%%%%%%%%%%%
\title{CME 338 Final Project: USYMLQ and USYMQR}
\date{\today}
\author{Jui-Hsien Wang}
\maketitle
%%%%%%%%%%%%%%%%%%%%%%%%%%%%%%%%%%%%%%%%%%%%%%%%%%%%%%%%%%%%%%%%%%%%%%%
%% Body
%%%%%%%%%%%%%%%%%%%%%%%%%%%%%%%%%%%%%%%%%%%%%%%%%%%%%%%%%%%%%%%%%%%%%%%
\section{Introduction}


%%%%%%%%%%%%%%%%%%%%%%%%%%%%%%%%%%%%%%%%%%%%%%%%%%%%%%%%%%%%%%%%%%%%%%%
\section{USYMLQ Implementation} 

% tridiagonalization
% LQ factorization
% initial condition
% termination condition
% algorithm


%%%%%%%%%%%%%%%%%%%%%%%%%%%%%%%%%%%%%%%%%%%%%%%%%%%%%%%%%%%%%%%%%%%%%%%
\section{USYMQR Implementation} 

% tridiagonalization
% QR factorization
% initial condition
% termination condition
% algorithm


%%%%%%%%%%%%%%%%%%%%%%%%%%%%%%%%%%%%%%%%%%%%%%%%%%%%%%%%%%%%%%%%%%%%%%%
\section{Results} 

%%% NEED THESE %%%
% random square matrices 
% matrices with specific eigenvalues 
% matrices with specific singular values 
% ill-conditioned matrices 
% sparse matrices
% inconsistent/consistent least square matrices 

%%% MAYBE %%%
% comparison with lsqr and lsmr? 
% comparison with the naive solve


%%%%%%%%%%%%%%%%%%%%%%%%%%%%%%%%%%%%%%%%%%%%%%%%%%%%%%%%%%%%%%%%%%%%%%%
\section{Conclusion} 


%%%%%%%%%%%%%%%%%%%%%%%%%%%%%%%%%%%%%%%%%%%%%%%%%%%%%%%%%%%%%%%%%%%%%%%
%%%%%%%%%%%%%%%%%%%%%%%%%%%%%%%%%%%%%%%%%%%%%%%%%%%%%%%%%%%%%%%%%%%%%%%
\end{document}

% \begin{algorithm}[H]
% \caption{Backtracking Algorithm.}
% \label{Alg:Backtracking_Algorithm}
% \begin{algorithmic}[1]
%     \REQUIRE $f(x)$, maximum step length $\bar{\alpha} > 0$, 
%     \STATE $\alpha \gets \bar{\alpha}$;
%     \REPEAT
%     \STATE $\alpha \gets \rho \alpha$; 
%     \UNTIL $f(x_k + \alpha p_k) \leq f(x_k) + c \alpha \nabla f_k^T p_k$; 
%     \RETURN $\alpha_k = \alpha$.
% \end{algorithmic}
% \end{algorithm}

% \begin{figure}[H]
%     \centering
%     \includegraphics[width=4.0in]{images/Ellipse_illustration.pdf}
%     \caption{Illustration of the ellipse covering problem}
%     \label{fig:Ellipse_illustration}
% \end{figure}

